
%% 
%% Copyright 2007, 2008, 2009 Elsevier Ltd
%% 
%% This file is part of the 'Elsarticle Bundle'.
%% ---------------------------------------------
%% 
%% It may be distributed under the conditions of the LaTeX Project Public
%% License, either version 1.2 of this license or (at your option) any
%% later version.  The latest version of this license is in
%%    http://www.latex-project.org/lppl.txt
%% and version 1.2 or later is part of all distributions of LaTeX
%% version 1999/12/01 or later.
%% 
%% The list of all files belonging to the 'Elsarticle Bundle' is
%% given in the file `manifest.txt'.
%% 

%% Template article for Elsevier's document class `elsarticle'
%% with numbered style bibliographic references
%% SP 2008/03/01

\documentclass[preprint,12pt]{elsarticle}

%% Use the option review to obtain double line spacing
%% \documentclass[authoryear,preprint,review,12pt]{elsarticle}

%% Use the options 1p,twocolumn; 3p; 3p,twocolumn; 5p; or 5p,twocolumn
%% for a journal layout:
%% \documentclass[final,1p,times]{elsarticle}
%% \documentclass[final,1p,times,twocolumn]{elsarticle}
%% \documentclass[final,3p,times]{elsarticle}
%% \documentclass[final,3p,times,twocolumn]{elsarticle}
%% \documentclass[final,5p,times]{elsarticle}
%% \documentclass[final,5p,times,twocolumn]{elsarticle}

%% For including figures, graphicx.sty has been loaded in
%% elsarticle.cls. If you prefer to use the old commands
%% please give \usepackage{epsfig}

%% The amssymb package provides various useful mathematical symbols
\usepackage{amssymb}
%% The amsthm package provides extended theorem environments
%% \usepackage{amsthm}

%% The lineno packages adds line numbers. Start line numbering with
%% \begin{linenumbers}, end it with \end{linenumbers}. Or switch it on
%% for the whole article with \linenumbers.
%% \usepackage{lineno}

\journal{Computers \& Education}

\begin{document}

\begin{frontmatter}

%% Title, authors and addresses

%% use the tnoteref command within \title for footnotes;
%% use the tnotetext command for theassociated footnote;
%% use the fnref command within \author or \address for footnotes;
%% use the fntext command for theassociated footnote;
%% use the corref command within \author for corresponding author footnotes;
%% use the cortext command for theassociated footnote;
%% use the ead command for the email address,
%% and the form \ead[url] for the home page:
%% \title{Title\tnoteref{label1}}
%% \tnotetext[label1]{}
%% \author{Name\corref{cor1}\fnref{label2}}
%% \ead{email address}
%% \ead[url]{home page}
%% \fntext[label2]{}
%% \cortext[cor1]{}
%% \address{Address\fnref{label3}}
%% \fntext[label3]{}

\title{Profiling commodity computers for curricular activities in the natural sciences}

%% use optional labels to link authors explicitly to addresses:
%% \author[label1,label2]{}
%% \address[label1]{}
%% \address[label2]{}

\author{Jacob Fosso Tande}

\address{Colby College 4000, Mayflower Hill
Waterville, Maine 04901}

\begin{abstract}
%% Text of abstract
Advances in processor design and desktop application development have made Desktop 
computers versertile in an unprecedented fashion. The increase versatility includes 
unprecedented use in curricular activities accross disciplines. In a bid to understand 
how the versatility can be sustain for curricular activities, we profile iMac computers 
with {\it Mac OS X} operating system. We found that budgeting constraints and 
techbnological advances in hardware design leads to inventory overlap for old and new 
hardware. Data analysis also illustrate the need to match computer specifications with 
the type of activity expected to be carried out on it. This profiling study provides 
basis for baseline reference of desktop computers for research or curricular activities.
Furthermore, the profiling shows that computer activities slowdown, as perceived by users,
is intrinsict to the available computer resources. 
\end{abstract}

\begin{keyword}
Profiling \sep Desktop \sep Application \sep Performance \sep Computers \sep Curriculum
%% keywords here, in the form: keyword \sep keyword

%% PACS codes here, in the form: \PACS code \sep code

%% MSC codes here, in the form: \MSC code \sep code
%% or \MSC[2008] code \sep code (2000 is the default)

\end{keyword}

\end{frontmatter}

%% \linenumbers

%% main text
\section{Introduction}
\label{introduction}
%%(types of hardware used and why they are used, highlight the diversity)
Computers in the natural sciences (computer sciences, mathematics, physics, 
chemistry, biology, astronomy and geology) have rapidly taken a center role 
in research and curricular activities. The omnipresence of this essential instrument
is illustrated by how often they are used in classrooms as well as in research 
and teaching laboratories. In fact, almost any laboratory instrument, that 
you can think of in the natural sciences, has a computer interface. Desired computer 
features are defined by the type of activities that it is expected to perform. These activities 
range from high-throughput, through high-performance~\cite{FienenMN2015} and 
many-task~\cite{Jik-SooK2013} to commodity computing. Because the focus of this 
article is on curricular activities, commodity computing~\cite{BachthalerS2007}
 will be the basis for profiling. Also, in this article, our references to computers 
will imply {\it desktop or personal computer}, as is widely known. Furthermore, focus 
will only be on the central processing units~(microprocessor), the random access 
memory~(main memory) and the disk space~(permanent memory) because computer 
structure is beyond the scope of this work.
 

%(Evolution in processors and impact on Desktop applications)
One of the most central part of any modern computer device is the microprocessor.
It intergrates the central processing unit (CPU), the portion of a computer responsible
for performing the necessary arithmetic and logic operations, onto a single integrated 
circuit that controls the timing and general operation of the complete system. 
Microprocessors have a rich history~\cite{PachghareAA2013} marked by a spurr in innovation 
and increase in competition to further innovate~\cite{MarkoffJ1996,GoettlerR2011}. The power of a given microprocessor is, 
measured in bits, the most basic unit of coded instructions, expressed in a string of binary 
ones and zeros, which the computer interprets to carry out tasks. The more powerful the 
processor, the more instructions it can carry out at one time, leading to faster processing 
and more effectiveness at complex tasks. The power of the microprocessor has consistently 
doubled since the first microprocessor~\cite{intel4004}, growing from 4-bit chips to the widely used 
64-bit chips~\cite{BorkarS2011}. Computer designers and application developers have taken advantage of 
the microprocessor power to develop functionaly adapted computers to satisfy consumers 
needs and demands. This fast pace of change is perceivable in operating systems~(OS)~\cite{MaccabeA2006}, 
system software that manages communication between hardware and software, as they are 
being designed to be highly responsive. Applications are in turn developped to take 
advantage of the OS responsiveness. Because applications can be executed efficiently 
and reliably, the last decade has seen a big increase in the number of new applications~\cite{AnttonenM2011}, 
making the destop computers front and center in our daily activities including corricula 
activities.

%(compare and contrast research computing and curricular computing)
For a long time, computers have been used in academic research~\cite{KemalA2008}. They are largely being 
extended into the classroom as a result of the growth in computer power and application 
development. Computers constitute the data collection and analysis interface between the 
user and the instrument. Pedagogical experiments in the wet laboratories have evolved 
from manual to semi-digital~(some portion of the experiment is manual) and in some cases 
fully digital. Courses once thought to be abstract, largely based on the laws of physics 
and mathematics, now have laboratory sessions, with computer simulations carried out on 
pre-developed computer models. The emergence of data intensive fields of research~\cite{TenopirC2015,CarolTenopir2014} 
(Virtual reality, Artificial intelligence and everything computational such as biology, 
chemistry, physics, ecology, astronomy, fuild dynamics ${\ldots}$) has prompted the design
of new courses to accommodate demand. There is a number of web-based platforms~\cite{pearson,hawkes}, set up to 
dispense or to suplement lessons and in some cases they constitute a shared platform designed 
to give students similar experience as they proceed with curricular activities set up on the 
platform. Decision makers at colleges and universities have begun to consider user 
support for computers and softwares used for curricular activities. Professionals and staff 
scientists providing support, most often have limitted reference information on computer features 
for curricular activities, as well as on how to provide services with a user-experience~\cite{WilliamJD1988} 
of great quality. In some instances, reference is taken, solely, from research computing, which 
can lead to unintended consequences, such as the puchasing of computers with features well 
above or below the minimum requirement. Computer performance~\cite{LiljaJD2004} in research computing is 
closely linked to optimization (hardware and software for effectiveness, efficiency, reliability and 
consistency). This is very different from computing in curricular activities where, computer 
performance is defined by user perception and experience-\cite{FlautnerK2000}. Relying, solely, on 
information from research computing to determine computer features for curricular activities can 
lead to decissions that affect performance in ways not intended. It is also worth nothing that 
having experties in computer sciences and being knowledgeable about the type of activities 
to be carried out in the classrooms and laboratories is important in making informed 
decissions. In this work, we profile hundreds of computers used for curricular activities 
in the natural sciences. Data from profiling is analysed to highlight features and extract 
patterns to help inform purchasing agents on the selectoin of computer features for upgrade. 
The results will also serve as based line to new and experienced faculty when selecting 
features for computers requests or upgrades intended for currucular and research activities.


\section{Procedure}
\label{procedure}
More than 250 computers were profiled for system resource allocation 
and utilization. The profiling was carried out for both local and network 
accounts. For the two types of accounts, I profiled the systems at login 
after they have been idle or restarted. In other to reproduce end-user 
experience, the profiling script was set up to lauch at login, taking 
continous snapshots for 35 minutes.  The subprocess class and its Popen 
constructor in Python (version 3.6)~\cite{python366} was used to start, time, 
and exit the top command[citation]. The profiled computers had either the 
Mac OS X 10.8 Mountain Lion, the OS X 10.9 Mavericks, the OS X 10.11 El 
Captain or the macOS 10.12 Sierra operating system. We profiled computers 
from one classroom to the other, lebelling the log files to account for 
the date of data collection, the classroom and the computer inventory tag. 
A trunck directory and subdirectories were then created with each subdirectory 
named after the classroom from which profile data was collected. A copy of the 
trunck directory was saved for reference and future need before the log files 
were cleaned up and computer resource metrics extracted.
 

\section{Results}
\label{results}
Your Macintoch computer can perform only as well as its weakest link. The 
operating system and the hardware on which it runs are often limiting factors.
The disk size, the available memory and CPU resources, the network, and the 
active directory that links them all can limit the end-user's experience. 
To determine the limiting factors on classroom computers, data about system's 
(per minute) load average, the number of processes, total number of 
threads, number of sleeping processes, number of stuck processes, reads and 
writes to disk as well as disk usage by each user was extracted. Profile 
plots of the extracted data was made with particular attention to computers 
with known history of low end-user peceived performance.
 
%Thus, you need to choose your hardware carefully, and configure the hardware 
%and operating system appropriately. For example, if your workload is 
%I/O-bound, one approach is to design your application to minimize MySQL’s I/O
% workload. However, it’s often smarter to upgrade the I/O subsystem, install
% more memory, or reconfigure existing disks.

\subsection{What limits Macintoch Computer Performance?}
This is probably the question most Academic IT professionals and support 
center get asked alot. Answers to this question will surely depend on what 
activity the Macintoch is being used for. To answer the question, I carry out 
a system resource quided analyses of the profile data collected.

\subsubsection{System load average }

\subsubsection{Number of processes}

\subsubsection{Total number of threads}

\subsubsection{Number of sleeping processes}

\subsubsection{Number of stuck processes}

\subsubsection{Reads and Writes to disk}
 

\subsubsection{Disk usage by each user}
 

\section{Discussion}
\label{discussion}


\section{Conclusion}
\label{conclusion}


%% The Appendices part is started with the command \appendix;
%% appendix sections are then done as normal sections
%% \appendix

%% \section{}
%% \label{}

%% If you have bibdatabase file and want bibtex to generate the
%% bibitems, please use
%%
%%  \bibliographystyle{elsarticle-num} 
%%  \bibliography{<your bibdatabase>}

%% else use the following coding to input the bibitems directly in the
%% TeX file.
\begin{thebibliography}{00}

%% \bibitem{label}
%% Text of bibliographic item

\bibitem{FienenMN2015}Fienen M. N. ,Hunt J. R. (2015). High-Throughput Computing Versus 
High-Performance Computing for Groundwater Applications. Groundwater Technical Commentary. 53, 180-184.
\bibitem{Jik-SooK2013}Jik-Soo K., Sangwan K., Seokkyoo K., Seoyoung K., Seungwoo R.,Ok-Hwan B., and Soonwook H.
(2013). From High-Throughput Computing to Many-Task Computing: Challenges, Systems, and Applications. Proceedings,
 The 2nd International Conference on Software Technology. 19, 199-202
\bibitem{BachthalerS2007}Bachthaler S., Belli F. and Fedorova A. (2007). Desktop Workload Characterization for CMP/SMT and 
Implications for Operating System Design, In Proceedings of the Workshop on the Interaction between Operating 
Systems and Computer Architecture (WIOSCA), in conjunction with ISCA-34 
\bibitem{PachghareAA2013} Pachghare A.A., Andurkar G. K., Kulkarni A. M. (2013). A review on microprocessor and microprocessor specification.
International Journal of Science, Engineering and Technology Research (IJSETR). 2, 449-454
\bibitem{MarkoffJ1996} Markoff J. (December 1996). The Microprocessor's Impact on Society. Journal IEEE Micro. 16, 54-59
\bibitem{GoettlerR2011} Goettler, R. and Gordon, B. (December 2011). Competition and innovation in the microprocessor industry: Does AMD spur Intel to innovate more. journal of political economy. 119, 1141-1200.
\bibitem{intel4004} https://www.intel.co.uk/content/www/uk/en/history/museum-story-of-intel-4004.html. June 29, 2018
\bibitem{python366} https://www.python.org/about/. October 10, 2018
\bibitem{BorkarS2011} Borkar S. and Chien A. A. (May 2011). The Future of Microprocessors. Communications of the ACM. 54, 67-77
\bibitem{MaccabeA2006} Maccabe A., Bridges P., Brightwell R. and Riesen R. (2006). Recent Trends in Operating Systems and their Applicability to HPC. Cray User Group (CUG). O,1-9
\bibitem{AnttonenM2011} Anttonen M., Salminen A., Mikkonen T. and Taivalsaari A. (2011). Transforming the
 web into a real application platform: new technologies, emerging trends and missing pieces. Proceedings 
of the 2011 ACM Symposium on Applied Computing. SAC '11, 800-807
\bibitem{KemalA2008} Kemal A. Delic and Martin Anthony Walker.(2008). Emergence of the Academic Computing Clouds. 
Ubiquity. 9,
\bibitem{TenopirC2015} Tenopir, Carol, Dane Hughes, Suzie Allard, Mike Frame, Ben Birch, Lynn Baird, 
Robert Sandusky, Madison Langseth, and Andrew Lundeen. (December, 2015). Research Data Services in 
Academic Libraries:Data Intensive Roles for the Future? Journal of eScience Librarianship. 4, 1-21
\bibitem{CarolTenopir2014} Carol Tenopir, Robert J.Sandusky, Suzie Allard, Ben Birch. (May, 2014). 
Research data management services in academic research libraries and perceptions of librarians. 
Library \& Information Science Research. 36, 84-90
\bibitem{pearson} https://www.pearsonmylabandmastering.com . June 29, 2018
\bibitem{hawkes} http://www.hawkeslearning.com/ . June 29, 2018
\bibitem{WilliamJD1988} William J. Doll and Gholamreza Torkzadeh. (June, 1988). The Measurement of 
End-User Computing Satisfaction. MIS Quarterly, 12, 259-274.
\bibitem{LiljaJD2004} Lilja J. D. (2004). Measuring Computer Performance: Apractitioner's Guide (2nd ed.). 
Cambridge, Cambridge University Press. 
\bibitem{FlautnerK2000}Flautner K., Uhlig R., Mudge T. (2000). Thread-level Parallelism and Interactive
Performance of Desktop Applications. ASPLOS: Architectural Support for Programming Languages and Operating 
Systems. 2000, 129-138
\bibitem{}
\bibitem{}
\bibitem{}
\bibitem{}
\bibitem{}
\end{thebibliography}


\end{document}
\endinput

