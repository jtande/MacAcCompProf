\begin{thebibliography}{00}

%% \bibitem{label}
%% Text of bibliographic item

\bibitem{FienenMN2015}Fienen M. N. ,Hunt J. R. (2015). High-Throughput Computing Versus 
High-Performance Computing for Groundwater Applications. Groundwater Technical Commentary. 53, 180-184.
\bibitem{Jik-SooK2013}Jik-Soo K., Sangwan K., Seokkyoo K., Seoyoung K., Seungwoo R.,Ok-Hwan B., and Soonwook H.
(2013). From High-Throughput Computing to Many-Task Computing: Challenges, Systems, and Applications. Proceedings,
 The 2nd International Conference on Software Technology. 19, 199-202
\bibitem{BachthalerS2007}Bachthaler S., Belli F. and Fedorova A. (2007). Desktop Workload Characterization for CMP/SMT and 
Implications for Operating System Design, In Proceedings of the Workshop on the Interaction between Operating 
Systems and Computer Architecture (WIOSCA), in conjunction with ISCA-34 
\bibitem{PachghareAA2013} Pachghare A.A., Andurkar G. K., Kulkarni A. M. (2013). A review on microprocessor and microprocessor specification.
International Journal of Science, Engineering and Technology Research (IJSETR). 2, 449-454
\bibitem{MarkoffJ1996} Markoff J. (December 1996). The Microprocessor's Impact on Society. Journal IEEE Micro. 16, 54-59
\bibitem{GoettlerR2011} Goettler, R. and Gordon, B. (December 2011). Competition and innovation in the microprocessor industry: Does AMD spur Intel to innovate more. journal of political economy. 119, 1141-1200.
\bibitem{intel4004} https://www.intel.co.uk/content/www/uk/en/history/museum-story-of-intel-4004.html. June 29, 2018
\bibitem{python366} https://www.python.org/about/. October 10, 2018
\bibitem{BorkarS2011} Borkar S. and Chien A. A. (May 2011). The Future of Microprocessors. Communications of the ACM. 54, 67-77
\bibitem{MaccabeA2006} Maccabe A., Bridges P., Brightwell R. and Riesen R. (2006). Recent Trends in Operating Systems and their Applicability to HPC. Cray User Group (CUG). O,1-9
\bibitem{AnttonenM2011} Anttonen M., Salminen A., Mikkonen T. and Taivalsaari A. (2011). Transforming the
 web into a real application platform: new technologies, emerging trends and missing pieces. Proceedings 
of the 2011 ACM Symposium on Applied Computing. SAC '11, 800-807
\bibitem{KemalA2008} Kemal A. Delic and Martin Anthony Walker.(2008). Emergence of the Academic Computing Clouds. 
Ubiquity. 9,
\bibitem{TenopirC2015} Tenopir, Carol, Dane Hughes, Suzie Allard, Mike Frame, Ben Birch, Lynn Baird, 
Robert Sandusky, Madison Langseth, and Andrew Lundeen. (December, 2015). Research Data Services in 
Academic Libraries:Data Intensive Roles for the Future? Journal of eScience Librarianship. 4, 1-21
\bibitem{CarolTenopir2014} Carol Tenopir, Robert J.Sandusky, Suzie Allard, Ben Birch. (May, 2014). 
Research data management services in academic research libraries and perceptions of librarians. 
Library \& Information Science Research. 36, 84-90
\bibitem{pearson} https://www.pearsonmylabandmastering.com . June 29, 2018
\bibitem{hawkes} http://www.hawkeslearning.com/ . June 29, 2018
\bibitem{WilliamJD1988} William J. Doll and Gholamreza Torkzadeh. (June, 1988). The Measurement of 
End-User Computing Satisfaction. MIS Quarterly, 12, 259-274.
\bibitem{LiljaJD2004} Lilja J. D. (2004). Measuring Computer Performance: Apractitioner's Guide (2nd ed.). 
Cambridge, Cambridge University Press. 
\bibitem{FlautnerK2000}Flautner K., Uhlig R., Mudge T. (2000). Thread-level Parallelism and Interactive
Performance of Desktop Applications. ASPLOS: Architectural Support for Programming Languages and Operating 
Systems. 2000, 129-138
\bibitem{}
\bibitem{}
\bibitem{}
\bibitem{}
\bibitem{}
\end{thebibliography}
