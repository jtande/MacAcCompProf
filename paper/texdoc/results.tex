\section{Results}
\label{results}
Your Macintoch computer can perform only as well as its weakest link. The 
operating system and the hardware on which it runs are often limiting factors.
The disk size, the available memory and CPU resources, the network, and the 
active directory that links them all can limit the end-user's experience. 
To determine the limiting factors on classroom computers, data about system's 
(per minute) load average, the number of processes, total number of 
threads, number of sleeping processes, number of stuck processes, reads and 
writes to disk as well as disk usage by each user was extracted. Profile 
plots of the extracted data was made with particular attention to computers 
with known history of low end-user peceived performance.
 
%Thus, you need to choose your hardware carefully, and configure the hardware 
%and operating system appropriately. For example, if your workload is 
%I/O-bound, one approach is to design your application to minimize MySQL’s I/O
% workload. However, it’s often smarter to upgrade the I/O subsystem, install
% more memory, or reconfigure existing disks.

\subsection{What limits Macintoch Computer Performance?}
This is probably the question most Academic IT professionals and support 
center get asked alot. Answers to this question will surely depend on what 
activity the Macintoch is being used for. To answer the question, I carry out 
a system resource quided analyses of the profile data collected.

\subsubsection{System load average }

\subsubsection{Number of processes}

\subsubsection{Total number of threads}

\subsubsection{Number of sleeping processes}

\subsubsection{Number of stuck processes}

\subsubsection{Reads and Writes to disk}
 

\subsubsection{Disk usage by each user}
