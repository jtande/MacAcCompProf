\section{Introduction}
\label{introduction}
%%(types of hardware used and why they are used, highlight the diversity)
Computers in the natural sciences (computer sciences, mathematics, physics, 
chemistry, biology, astronomy and geology) have rapidly taken a center role 
in research and curricular activities. The omnipresence of this essential instrument
is illustrated by how often they are used in classrooms as well as in research 
and teaching laboratories. In fact, almost any laboratory instrument, that 
you can think of in the natural sciences, has a computer interface. Desired computer 
features are defined by the type of activities that it is expected to perform. These activities 
range from high-throughput, through high-performance~\cite{FienenMN2015} and 
many-task~\cite{Jik-SooK2013} to commodity computing. Because the focus of this 
article is on curricular activities, commodity computing~\cite{BachthalerS2007}
 will be the basis for profiling. Also, in this article, our references to computers 
will imply {\it desktop or personal computer}, as is widely known. Furthermore, focus 
will only be on the central processing units~(microprocessor), the random access 
memory~(main memory) and the disk space~(permanent memory) because computer 
structure is beyond the scope of this work.
 

%(Evolution in processors and impact on Desktop applications)
One of the most central part of any modern computer device is the microprocessor.
It intergrates the central processing unit (CPU), the portion of a computer responsible
for performing the necessary arithmetic and logic operations, onto a single integrated 
circuit that controls the timing and general operation of the complete system. 
Microprocessors have a rich history~\cite{PachghareAA2013} marked by a spurr in innovation 
and increase in competition to further innovate~\cite{MarkoffJ1996,GoettlerR2011}. The power of a given microprocessor is, 
measured in bits, the most basic unit of coded instructions, expressed in a string of binary 
ones and zeros, which the computer interprets to carry out tasks. The more powerful the 
processor, the more instructions it can carry out at one time, leading to faster processing 
and more effectiveness at complex tasks. The power of the microprocessor has consistently 
doubled since the first microprocessor~\cite{intel4004}, growing from 4-bit chips to the widely used 
64-bit chips~\cite{BorkarS2011}. Computer designers and application developers have taken advantage of 
the microprocessor power to develop functionaly adapted computers to satisfy consumers 
needs and demands. This fast pace of change is perceivable in operating systems~(OS)~\cite{MaccabeA2006}, 
system software that manages communication between hardware and software, as they are 
being designed to be highly responsive. Applications are in turn developped to take 
advantage of the OS responsiveness. Because applications can be executed efficiently 
and reliably, the last decade has seen a big increase in the number of new applications~\cite{AnttonenM2011}, 
making the destop computers front and center in our daily activities including corricula 
activities.

%(compare and contrast research computing and curricular computing)
For a long time, computers have been used in academic research~\cite{KemalA2008}. They are largely being 
extended into the classroom as a result of the growth in computer power and application 
development. Computers constitute the data collection and analysis interface between the 
user and the instrument. Pedagogical experiments in the wet laboratories have evolved 
from manual to semi-digital~(some portion of the experiment is manual) and in some cases 
fully digital. Courses once thought to be abstract, largely based on the laws of physics 
and mathematics, now have laboratory sessions, with computer simulations carried out on 
pre-developed computer models. The emergence of data intensive fields of research~\cite{TenopirC2015,CarolTenopir2014} 
(Virtual reality, Artificial intelligence and everything computational such as biology, 
chemistry, physics, ecology, astronomy, fuild dynamics ${\ldots}$) has prompted the design
of new courses to accommodate demand. There is a number of web-based platforms~\cite{pearson,hawkes}, set up to 
dispense or to suplement lessons and in some cases they constitute a shared platform designed 
to give students similar experience as they proceed with curricular activities set up on the 
platform. Decision makers at colleges and universities have begun to consider user 
support for computers and softwares used for curricular activities. Professionals and staff 
scientists providing support, most often have limitted reference information on computer features 
for curricular activities, as well as on how to provide services with a user-experience~\cite{WilliamJD1988} 
of great quality. In some instances, reference is taken, solely, from research computing, which 
can lead to unintended consequences, such as the puchasing of computers with features well 
above or below the minimum requirement. Computer performance~\cite{LiljaJD2004} in research computing is 
closely linked to optimization (hardware and software for effectiveness, efficiency, reliability and 
consistency). This is very different from computing in curricular activities where, computer 
performance is defined by user perception and experience-\cite{FlautnerK2000}. Relying, solely, on 
information from research computing to determine computer features for curricular activities can 
lead to decissions that affect performance in ways not intended. It is also worth nothing that 
having experties in computer sciences and being knowledgeable about the type of activities 
to be carried out in the classrooms and laboratories is important in making informed 
decissions. In this work, we profile hundreds of computers used for curricular activities 
in the natural sciences. Data from profiling is analysed to highlight features and extract 
patterns to help inform purchasing agents on the selectoin of computer features for upgrade. 
The results will also serve as based line to new and experienced faculty when selecting 
features for computers requests or upgrades intended for currucular and research activities.
